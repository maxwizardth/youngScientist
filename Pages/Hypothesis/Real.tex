\documentclass[45pt]{article}
\usepackage{blindtext}

% Language setting
% Replace `english' with e.g. `spanish' to change the document language
\usepackage[english]{babel}

% Set page size and margins
% Replace `letterpaper' with `a4paper' for UK/EU standard size
\usepackage[letterpaper,total={6in,8in},top=2cm, bottom=2cm, left=3cm,right=3cm,marginparwidth=1.75cm]{geometry}
\usepackage{mathptmx}
\usepackage[10pt]{moresize}
% Useful packages
\usepackage{amsmath,amssymb}
\usepackage{graphicx}
\usepackage[colorlinks=true, allcolors=blue]{hyperref}
\usepackage{amsfonts}
\usepackage{mathptmx}
\usepackage{anyfontsize}
\usepackage{t1enc}

\title{GROUP 1}
\author{Project 1}

\begin{document}
\begin{titlepage}
    \begin{center}
        \vspace*{1cm}
        \Huge
        \textbf{Linear Algebra}
        
        \vspace{0.5cm}
        \LARGE
        Vector Space
        
        \vspace{0.5cm}
        {\fontsize{70}{70}\selectfont Group 1}
        \vspace{0.6cm}        
            
        A project providing solution to  different problems on \\vector Space.
        \vspace{0.8cm}
            
        \includegraphics[width=0.4\textwidth]{UI logo.png}

        \vspace{1.0cm}
        
        \Large{
         Mathematics (BSc. and B.Ed)\\
        University of Ibadan\\
        Nigeria.\\
        19/04/2023}
            
    \end{center}
\end{titlepage}
\Huge{\centering{Project 1}} 
\\\\
\Large{\pmb{Questions}}
\begin{enumerate}
  \item Prove that 

  
  $ Y \pmb{=} \begin{Bmatrix}
    \begin{pmatrix} 
    & a  \\
    & b & \\
    & c & \\
    & d & \\
    \end{pmatrix} \in \mathbb{R}^4 : ~
    a-b+3d\pmb{=}0
    \end{Bmatrix}$
  
is a subspace of $\mathbb{R}^4\\\\$
  
  \item Prove that 
$X \pmb{=} \begin{Bmatrix}
    A \in M_n(\mathbb{R}): ~ A=A^T
    \end{Bmatrix}$ is a subspace of $ M_n(\mathbb{R})$
\item Prove that
$ \mathcal{C} \subset M_n(\mathbb{R}) ~ of ~ skewed~ symetric$ \\
$\mathcal{C} \pmb{=} \begin{Bmatrix}
    B \in M_n(\mathbb{R}): ~ B=-B^T
	\end{Bmatrix}$
is a subspace of $ M_n(\mathbb{R})$
 \item Show that the set  $\mathcal{B}\subset M_n(\mathbb{F})$ of upper triangular matrices is a subspace of $M_n(\mathbb{F})$ i.e $\mathcal{B}=\{ D \in M_n(\mathbb{R}): a_{ij}=0$ if $ i>j \}$

\end{enumerate}
$ $\\

\Large{\pmb{Solutions}}
\begin{enumerate}
    \item Since ~ 0-0+3*0=0, then Y has zero element i.e \\ \\
$ there ~ exist \begin{pmatrix}
    & 0  \\
    & 0 & \\
    & 0 & \\
    & 0 & \\
    \end{pmatrix}
    \in Y
      $ \\\\
\pmb{Hence Y is not empty.} \\ \\ 

choose $x, y \in Y$ then \\\\
$x = \begin{pmatrix}
    & x_1  \\
    & x_2 & \\
    & x_3 & \\
    & x_4 & \\
    \end{pmatrix} ~ and ~ y=  \begin{pmatrix}
    & y_1  \\
    & y_2 & \\
    & y_3 & \\
    & y_4 & \\
    \end{pmatrix} \\\\
    where ~ x_1-x_2+3x_4=0 ~ and ~ y_1-y_2+3y_4=0
      $ \\\\

choose $\alpha \in \mathbb{F}$ where $\mathbb{F}$ is a Field then, \\\\
x+$\alpha y= \begin{pmatrix}
    & x_1  \\
    & x_2 & \\
    & x_3 & \\
    & x_4 & \\
    \end{pmatrix} + \alpha \begin{pmatrix}
    & y_1  \\
    & y_2 & \\
    & y_3 & \\
    & y_4 & \\
    \end{pmatrix} =
    \begin{pmatrix}
    & x_1  \\
    & x_2 & \\
    & x_3 & \\
    & x_4 & \\
    \end{pmatrix} +
    \begin{pmatrix}
    & \alpha y_1  \\
    & \alpha y_2 & \\
    & \alpha y_3 & \\
    & \alpha y_4 & \\
    \end{pmatrix}\\\\$
    
    $ x+ \alpha y = 
    \begin{pmatrix}
    & x_1 + \alpha y_1  \\
    & x_2 + \alpha y_2& \\
    & x_3 + \alpha y_3& \\
    & x_4 + \alpha y_4& \\
    \end{pmatrix} $ \\\\\\\\
Now we need to check whether $(x+ \alpha y )$ is an element in Y or not.  To do that we will check for the condition necessary for element in Y.\\

$(x_1+\alpha y_1) - (x_2+\alpha y_2) +3(x_4+\alpha y_4)\\\\$ 
$x_1+\alpha y_1 - x_2-\alpha y_2 +3x_4+3\alpha y_4\\\\$
\pmb{Rearrange} \\\\
$x_1-x_2+3x_4+\alpha y_1-\alpha y_2+3\alpha y_4\\$
$x_1-x_2+3x_4+\alpha (y_1-y_2+3 y_4) = 0+3(0)=0\\\\$
This satisfied the condition necessary for element in Y.\\
So, $(x + \alpha y )\in Y$
Therefore, Y is a subspace of $\mathbb{R}^4\\$

\item Since all zero matrix of $M_n(\mathbb{R})$  is symmetric, then Y has zero element i.e \\ \\
$ there ~ exist \begin{pmatrix}
    & 0 & 0 & 0 &.&.&.& 0 \\
    & 0 & 0 & 0 &.&.&.& 0\\
    & 0 & 0 & 0 &.&.&.& 0\\
    & . & . & . &.&.&.& 0\\
    & . & . & . &.&.&.& 0\\
    & . & . & . &.&.&.& 0\\
    & 0 & 0 & 0 &.&.&.& 0\\
    \end{pmatrix} 
    \in X
      $ \\\\
\pmb{Hence X is not empty.} \\ \\ 

choose $A, B \in X $, since A and B are symmetric matrix     `then \\\\
$ a_{i,j}=a_{j,i} ~ and ~ b_{i,j}=b_{j,i} ~ (where ~ 1 \leq i \leq n ~and ~1 \leq j \leq n )\\$
$a_{ij} ~ and ~ b_{i,j}$~ denote~ entry~ in~ row-i ~column-j ~in ~A~ and ~B~ respectively

let say,

$A= \begin{pmatrix}
    & a_{1,1} & a_{1,2} & a_{1,3} &.&.&.& a_{1,n}  \\
    & a_{2,1} & a_{2,2} & a_{2,3} &.&.&.& a_{2,n}  \\
    & a_{3,1} & a_{3,2} & a_{3,3} &.&.&.& a_{3,n}  \\
    & . & . & . &.&.&.&.\\
    & . & . & . &.&.&.&.\\
    & . & . & . &.&.&.&.\\
    & a_{n,1} & a_{n,2} & a_{n,3} &.&.&.& a_{n,n}  \\
    \end{pmatrix}$, $
    B=\begin{pmatrix}
        & b_{1,1} & b_{1,2} & b_{1,3} &.&.&.& b_{1,n}  \\
        & b_{2,1} & b_{2,2} & b_{2,3} &.&.&.& b_{2,n}  \\
        & b_{3,1} & b_{3,2} & b_{3,3} &.&.&.& b_{3,n}  \\
        & . & . & . &.&.&.&.\\
        & . & . & . &.&.&.&.\\
        & . & . & . &.&.&.&.\\
        & b_{n,1} & b_{n,2} & b_{n,3} &.&.&.& b_{n,n} 
     \end{pmatrix}
    $\\\\


Now, let us simplify $A+ \alpha B$ (where $\alpha \in \mathbb{F}$ and $\mathbb{F}$ is a Field of Real number)\\\\
$A+ \alpha B =\begin{pmatrix}
    & a_{1,1} & a_{1,2} & a_{1,3} &.&.&.& a_{1,n}  \\
    & a_{2,1} & a_{2,2} & a_{2,3} &.&.&.& a_{2,n}  \\
    & a_{3,1} & a_{3,2} & a_{3,3} &.&.&.& a_{3,n}  \\
    & . & . & . &.&.&.&.\\
    & . & . & . &.&.&.&.\\
    & . & . & . &.&.&.&.\\
    & a_{n,1} & a_{n,2} & a_{n,3} &.&.&.& a_{n,n}
    \end{pmatrix}$ +
    $\alpha \begin{pmatrix}
        & b_{1,1} & b_{1,2} & b_{1,3} &.&.&.& b_{1,n}  \\
        & b_{2,1} & b_{2,2} & b_{2,3} &.&.&.& b_{2,n}  \\
        & b_{3,1} & b_{3,2} & b_{3,3} &.&.&.& b_{3,n}  \\
        & . & . & . &.&.&.&.\\
        & . & . & . &.&.&.&.\\
        & . & . & . &.&.&.&.\\
        & b_{n,1} & b_{n,2} & b_{n,3} &.&.&.& b_{n,n}  \\
     \end{pmatrix}
    $\\\\

    $
        A+ \alpha B =\begin{pmatrix}
            & a_{1,1} & a_{1,2} & a_{1,3} &.&.&.& a_{1,n}  \\
            & a_{2,1} & a_{2,2} & a_{2,3} &.&.&.& a_{2,n}  \\
            & a_{3,1} & a_{3,2} & a_{3,3} &.&.&.& a_{3,n}  \\
            & . & . & . &.&.&.&.\\
            & . & . & . &.&.&.&.\\
            & . & . & . &.&.&.&.\\
            & a_{n,1} & a_{n,2} & a_{n,3} &.&.&.& a_{n,n}  \\
            \end{pmatrix} + \begin{pmatrix}
        & \alpha b_{1,1} & \alpha b_{1,2} & \alpha b_{1,3} &.&.&.& \alpha b_{1,n}  \\
        & \alpha b_{2,1} & \alpha b_{2,2} & \alpha b_{2,3} &.&.&.& \alpha b_{2,n}  \\
        & \alpha b_{3,1} & \alpha b_{3,2} & \alpha b_{3,3} &.&.&.& \alpha b_{3,n}  \\
        & . & . & . &.&.&.&.\\
        & . & . & . &.&.&.&.\\
        & . & . & . &.&.&.&.\\
        & \alpha b_{n,1} & \alpha b_{n,2} & \alpha b_{n,3} &.&.&.& \alpha b_{n,n}  \\
        \end{pmatrix}\\\\
    $

$
    A+ \alpha B =
    \begin{pmatrix}
        
        & a_{1,1} \alpha + b_{1,1} & a_{1,2} + \alpha b_{1,2} & a_{1,3} + \alpha b_{1,3}  &.&.&.& a_{1,n} +\alpha b_{1,n}  \\
        & a_{2,1} \alpha + b_{2,1} & a_{2,2} + \alpha b_{2,2} & a_{2,3} + \alpha b_{2,3}  &.&.&.& a_{2,n} +\alpha b_{2,n}  \\
        & a_{3,1} \alpha + b_{3,1} & a_{3,2} + \alpha b_{3,2} & a_{3,3} + \alpha b_{3,3}  &.&.&.& a_{3,n} +\alpha b_{3,n}  \\
        & . & . & . &.&.&.&.\\
        & . & . & . &.&.&.&.\\
        & . & . & . &.&.&.&.\\
        & a_{n,1}+\alpha b_{n,1} & a_{n,2} + \alpha b_{n,2} & a_{n,3}+\alpha b_{n,3} &.&.&.& a_{n,n}+\alpha b_{n,n}  \\
    \end{pmatrix}\\\\$

Let C be a matrix such such that 
$C= A+ \alpha B \\$ then we see that 
$c_{i,j}=a_{i,j}+\alpha b_{i,j} \\$
($ c_{i,j},a_{i,j}~ and ~b_{i,j} $ is an entry in row i column j of matrix C, A and B respectively)

Now let's check if C is symmetric or not\\
$c_{i,j}=a_{i,j}+\alpha b_{i,j}\\ $
By symmetric property of A and B\\ $b_{i,j}=b_{j,i}$ and $a_{i,j}=a_{j,i}$\\
So, $c_{i,j}=a_{i,j}+\alpha b_{i,j} =a_{j,i}+\alpha b_{j,i}\\$ 
But $c_{j,i} =a_{j,i}+\alpha b_{j,i}\\$
This implies that $c_{i,j}=c_{j,i}\\$
showing that $C=C^T ~i.e~ A+ \alpha B$ is a symmetric matrix \\
Therefore $A+ \alpha B  \in X\\$
\pmb{Hence X is a subspace of $M_{n}$}


\item Since all zero matrix of $M_n(\mathbb{R}) $ are symmetrically skewed, then $\mathcal{C}$  has zero element\\
\pmb{Hence $\mathcal{C}$ is not empty.} \\ 

choose $A, B \in \mathcal{C} $, since A and B are symmetrically skewed matrix then 
$ a_{i,j}=-a_{j,i} ~ and ~ b_{i,j}=-b_{j,i} ~ (where ~ 1 \leq i \leq n ~and ~1 \leq j \leq n )\\$
$a_{ij}$ and $b_{i,j}$~ denote~ entry~ in~ row-i ~column-j ~in ~A~ and ~B~ respectively\\

Just as you have seen the addition of matrix in Question 2\\
Let C be a matrix such such that 
$C= A+ \alpha B $ then we see that \\
$c_{i,j}=a_{i,j}+\alpha b_{i,j} $
($ c_{i,j},a_{i,j}~ and ~b_{i,j} $,is an entry in row i column j of matrix C, A and B respectively)\\

\pmb{Now let's check if C is skew-symmetric or not}\\
$c_{i,j}=a_{i,j}+\alpha b_{i,j} $\\
from the property of A and B, we know that \\
$b_{i,j}=-b_{j,i}$ and $a_{i,j}=-a_{j,i}$\\
So, $c_{i,j}=a_{i,j}+\alpha b_{i,j} =-a_{j,i}+\alpha *-b_{j,i}$\\
$c_{i,j}=-a_{j,i}-\alpha *b_{j,i}=-(a_{j,i}+\alpha b_{j,i})$ \\
But $c_{j,i} =a_{j,i}+\alpha b_{j,i}$\\
So, $c_{i,j}=-c_{j,i}$\\
showing that $C=-C^T ~i.e $ $ (A+ \alpha B)$ is a skew-symmetric  matrix \\
Therefore $(A+ \alpha B)  \in \mathcal{C}$\\
\pmb{Hence $\mathcal{C}$ is a subspace of $M_n(\mathbb{R})$}


\item There exist Zero matrix of $M_n(\mathbb{F})$ that is upper triangular matrix therefore $\mathcal{B}$ is not empty\\\\
choose $A, B \in \mathcal{B} $, since A and B are upper triangular matrix then 

$ a_{i,j}=0 ~ and ~ b_{i,j}=0~ \forall_{i>j} ~ (where ~ 1 \leq i \leq n ~and ~1 \leq j \leq n )\\$
$a_{ij} ~ and ~ b_{i,j}$~ denote~ entry~ in~ row-i ~column-j ~in ~A~ and ~B~ respectively\\
Let C be a matrix such such that \\
$C= A+ \alpha B $ then we see that \\
$c_{i,j}=a_{i,j}+\alpha b_{i,j} $\\
($ c_{i,j},a_{i,j}~ and ~b_{i,j} $ is an entry in row-i column-j of matrix C, A and B respectively)\\

Now let's check if C is a upper triangular matrix or not\\
$c_{i,j}=a_{i,j}+\alpha b_{i,j} $\\
from the property of A and B,\\ we know that $b_{i,j}=0$ and $a_{i,j}=0 ~(when~ever ~i>j)$\\
So, $c_{i,j}=0+\alpha *0 ~(whenever~ i > j)$\\
$c_{i,j}=0+0 ~(whenever ~i>j)$ \\
$c_{i,j}=0 ~(whenever ~ i>j)$ \\\\
This shows that matrix C is an upper triangular matrix\\
Therefore, $ C \in \mathcal{C} $ i.e $(A+\alpha B) \in \mathcal{C}\\\\$
\pmb{Hence $\mathcal{C}$ is a subspace of $M_n(\mathbb{F})$}\\\\
\end{enumerate}

\Large{\pmb{CONTRIBUTORS}}
\begin{enumerate}
    \item 221375
    \item 221382
    \item 222634
    \item 222666
    \item 222667
    \item 222678
    \item 222681
    \item 222686
    \item 222691
    \item 223769






\section*{Project 3}
     Explore Pascal triangle, Give the meaning/interpretation of the number in the 4th ,5th and 6th diagonal in terms of dimension of vector space.
     
\section{Project 4}
    Let $T={O \in M_n(\mathbb{R}) :Tr(O)=0 }$ be a vector space of Traceless nxn matrices. compute a basis and dimension of T.
    
prove that the 
\LARGE{Solutions}
A Basis of Traceless matrices is below
$\{e_{1,1}, e_{1,2}, e_{1,3}...e_{1,n},e_{2,1}, e_{2,2}, e_{2,3}...e_{2,n}...e_{n,n}\}$\\

 where e_{1,1}= \begin{bmatrix}
              1 & 0 & 0 &.& .& .& 0\\
              0 & 0 & 0 &.& .& .& 0\\
              0 & 0 & 0 &.& .& .& 0\\
              
              0 & 0 & 0 &.& .& .& 0\\
            \end{bmatrix}



   
\section*{\Huge{Project 5}}
\Large{Questions}
\begin{enumerate}
  \item Prove that the set 
$ \\
\begin{Bmatrix}
  \begin{pmatrix}
    &1&\\&1&\\ &1&\\&1&\\
  \end{pmatrix},

  \begin{pmatrix}
    &1&\\&1&\\ &-1&\\&1&\\
  \end{pmatrix},
  \begin{pmatrix}
    &1&\\&-1&\\ &1&\\&-1&\\
  \end{pmatrix},
  \begin{pmatrix}
    &1&\\&1&\\ &0&\\&1&\\
  \end{pmatrix}
\end{Bmatrix}\\\\$
is a linear independent set of vectors in $\mathbb{R}^4$

\item Prove that 
$\begin{pmatrix} &1&\\&2&\\ &1&\\&1&\\ \end{pmatrix}$
can be expressed as linear combinations \\

\item Let $\{V_1, V_2, V_3, V_4\}$ be a linear independent family of vectors in $\mathbb{R}^4$ prove that 
$\{V_1+V_2,~ V_2+V_3,~ V_3+V_4,~ V_4+V_1\}$ is not linearly independent but $\{V_1 + V_2, V_2, V_3, V_4\}$ is linearly independent.\\$ $

\item Let f,g,h belong to the space of infinitely continuously differentiable real value function $C^{\infty} (\mathbb{R})$
 such that $f(x) = e^x,~ g(x)= e^{2x} ~and ~h(x)=e^{3x}$ show that f,g,h are linearly independent.
\end{enumerate}
$ $ \\
\Large{\pmb{Solutions}}
\renewcommand{\labelenumii}{\arabic{enumii}}
\begin{enumerate}
  \item To prove that 
$\begin{Bmatrix}
    \begin{pmatrix}
      &1&\\&1&\\ &1&\\&1&\\
    \end{pmatrix},
  
    \begin{pmatrix}
      &1&\\&1&\\ &-1&\\&1&\\
    \end{pmatrix},
    \begin{pmatrix}
      &1&\\&-1&\\ &1&\\&-1&\\
    \end{pmatrix},
    \begin{pmatrix}
      &1&\\&1&\\ &0&\\&1&\\
    \end{pmatrix}
  \end{Bmatrix}$ is a set of linearly dependent vectors we need to 
  show that there exist real numbers $\alpha _1,~ \alpha_2,~ \alpha_3~ and~ \alpha_4$, not all of them are zero, such that \\

  $\alpha_1 \begin{pmatrix}
    &1&\\&1&\\ &1&\\&1&\\
  \end{pmatrix} +
  \alpha_2\begin{pmatrix}
    &1&\\&1&\\ &-1&\\&1&\\
  \end{pmatrix}+
  \alpha_3 \begin{pmatrix}
    &1&\\&-1&\\ &1&\\&-1&\\
  \end{pmatrix}+
  \alpha_4 \begin{pmatrix}
    &1&\\&1&\\ &0&\\&1&\\
  \end{pmatrix} = \begin{pmatrix}
    &0&\\&0&\\ &0&\\&0&\\
  \end{pmatrix} $
 \\\\
\par \Large{\pmb{Proof:}}\\\\
$\alpha_1 \begin{pmatrix}
  &1&\\&1&\\ &1&\\&1&\\
\end{pmatrix} +
\alpha_2\begin{pmatrix}
  &1&\\&1&\\ &-1&\\&1&\\
\end{pmatrix}+
\alpha_3 \begin{pmatrix}
  &1&\\&-1&\\ &1&\\&-1&\\
\end{pmatrix}+
\alpha_4 \begin{pmatrix}
  &1&\\&1&\\ &0&\\&1&\\
\end{pmatrix} = \begin{pmatrix}
  &0&\\&0&\\ &0&\\&0&\\
\end{pmatrix}$\\\\

$ \begin{pmatrix}
  &\alpha_1&\\&\alpha_1&\\ &\alpha_1&\\&\alpha_1&\\
\end{pmatrix} +
\begin{pmatrix}
  &\alpha_2&\\&\alpha_2&\\ &-\alpha_2&\\&\alpha_2&
\end{pmatrix}+
\begin{pmatrix}
  &\alpha_3&\\&-\alpha_3\\ &\alpha_3&\\&-\alpha_3&\\
\end{pmatrix}+
\begin{pmatrix}
  &\alpha_4&\\&\alpha_4&\\ &0&\\&\alpha_4&\\
\end{pmatrix} =\begin{pmatrix}
  &0&\\&0&\\ &0&\\&0&\\
\end{pmatrix}$\\\\

$ \begin{pmatrix}
    &\alpha_1+\alpha_2+\alpha_3 +\alpha_4\\
    &\alpha_1+\alpha_2-\alpha_3 +\alpha_4\\
    &\alpha_1-\alpha_2+\alpha_3\\
    &\alpha_1+\alpha_2-\alpha_3 +\alpha_4\\
  \end{pmatrix} =\begin{pmatrix}
    &0&\\&0&\\ &0&\\&0&\\
  \end{pmatrix}$\\
      $\alpha_1+\alpha_2+\alpha_3 +\alpha_4 =0$\\
      $\alpha_1+\alpha_2-\alpha_3 +\alpha_4=0$\\
      $\alpha_1-\alpha_2+\alpha_3=0$\\
      $\alpha_1+\alpha_2-\alpha_3 +\alpha_4=0$\\
  
 \pmb{ Let us solve the equation.}\\
      $\alpha_1+\alpha_2+\alpha_3 +\alpha_4 =0...eq(1)$\\
      $\alpha_1+\alpha_2-\alpha_3 +\alpha_4=0...eq(2)$\\
      $\alpha_1-\alpha_2+\alpha_3=0...eq(3)$\\
      $\alpha_1+\alpha_2-\alpha_3 +\alpha_4=0...eq(4)$\\

\pmb{From eq(iii), $\alpha_2=\alpha_1+\alpha_3$...eq(5)}\\
 substitute $\alpha_2=\alpha_1+\alpha_3$ in to equation (1),(2) and (4).
    
$ \begin{matrix}
       \alpha_1+\alpha_1+\alpha_3+\alpha_3 +\alpha_4 =0\\
      \alpha_1+\alpha_1+\alpha_3-\alpha_3 +\alpha_4=0\\
      \alpha_1+\alpha_1+\alpha_3-\alpha_3 +\alpha_4=0
\end{matrix} $\\\\

$\begin{matrix}
     2\alpha_1+2\alpha_3 +\alpha_4 =0 ...eq(6)\\
     2\alpha_1+\alpha_4=0...eq(7)\\
     2\alpha_1+\alpha_4=0...eq(8)
\end{matrix}$
  \\\\
\pmb{subtract eq(7) from eq(6)}\\
$2\alpha_3 = 0 \implies \alpha_3=0$\\
Note: eq(7) and eq(8) are the same this shows that there will be infinitely many solutions.\\ 
So, choose $\alpha_1=1$, then\\
$2(1)+\alpha_4=0 \implies \alpha_4=-2$\\

But Recall that $\alpha_2=\alpha_1+\alpha3$\\
so, $\alpha_2= 1+0 = 1,$ \\
Therefore, $\alpha_1=1, \alpha_2=1, \alpha_3=0~ and ~\alpha_4=-2$\\
Since We find some real numbers ($\alpha_1, \alpha_2, \alpha_3~and ~\alpha_4$) that not all of them equal to Zero therefore the given set of vectors are \pmb{linearly dependent}.\\
    
\item \begin{enumerate}
    \item If set of vector {$V_1,V_2,V_3,V_4$} are linearly independent then \\
 $\alpha_1 V_1+\alpha_2 V_2+\alpha_3 V_3+ \alpha_4 V_4$ =
$\begin{pmatrix}
0\\0\\0\\0
\end{pmatrix}
$ only if $\alpha_1 = \alpha_2 =\alpha_3= \alpha_4=0$\\\\
Now let us consider set of vector ${V_1+V_2,V_2,V_3,V_4}$
we need to show that,\\

$\beta_1 (V_1+V_2)+\beta_2 V_2+\beta_3 V_3+ \beta_4 V_4$ =
$\begin{pmatrix}
0\\0\\0\\0
\end{pmatrix}
$ 
 only if $\beta_1 = \beta_2 =\beta_3= \beta_4=0$\\\\
 
\pmb{Proof:}\\
$\beta_1 (V_1+V_2)+\beta_2 V_2+\beta_3 V_3+ \beta_4 V_4$ =
$\begin{pmatrix}
0\\0\\0\\0
\end{pmatrix}
$ \\

$\beta_1 V_1+ \beta_1 V_2+\beta_2 V_2+\beta_3 V_3+ \beta_4 V_4$ =
$\begin{pmatrix}
0\\0\\0\\0
\end{pmatrix}
$ 

$\beta_1 V_1+ (\beta_1 +\beta_2) V_2+\beta_3 V_3+ \beta_4 V_4$ =
$\begin{pmatrix}
0\\0\\0\\0\\
\end{pmatrix}$

let $\beta_v=\beta_1+\beta_2$ then,\\

$\beta_1 V_1+ \beta_v  V_2+\beta_3 V_3+ \beta_4 V_4$ =
$\begin{pmatrix}
0\\0\\0\\0\\
\end{pmatrix}$\\
Since we know that $V_1,V_2,V_3,V_4$ are linearly independent then \\ $\beta_1=\beta_v=\beta_3=\beta_4=0$\\
Recall that $\beta_3=\beta_1+\beta_2$\\
so, $\beta_1+\beta_2=0$\\ 
since $\beta_1=0$ then, $0+\beta_2$=0\\  
therefore, $\beta_2=0$\\
Hence, $\beta_1=\beta_2=\beta_3=\beta_4=0$\\ 
Since , $\beta_1=\beta_2=\beta_3=\beta_4=0$\\ 
then the set of vector  {$V_1+V_2,V_2,V_3,V_4$} are linearly independent then \\

\item we need to show that there exist some real numbers such that not all are zero that satisfied below equation\\
$\beta_1 (V_1+V_2)+\beta_2 (V_2+V_3)+\beta_3 (V_3+V_4)+ \beta_4 (V_4+V_1)$ =
$\begin{pmatrix}
0\\0\\0\\0
\end{pmatrix}
$ \\\\

\pmb{Proof:}\\
$\beta_1 (V_1+V_2)+\beta_2 (V_2+V_3)+\beta_3 (V_3+V_4)+ \beta_4 (V_4+V_1)$ =
$\begin{pmatrix}
0\\0\\0\\0
\end{pmatrix}$\\
$\beta_1 V_1+\beta_1 V_2+\beta_2 V_2+ \beta_2 V_3+\beta_3 V_3+\beta_3 V_4+ \beta_4 V_4+ \beta_4 V_1$ =
$\begin{pmatrix}
0\\0\\0\\0
\end{pmatrix}$\\

$\beta_1 V_1+\beta_4 V_1+\beta_2 V_2+ \beta_1 V_2+\beta_3 V_3+\beta_2 V_3+ \beta_4 V_4+ \beta_3 V_4$ =
$\begin{pmatrix}
0\\0\\0\\0
\end{pmatrix}$\\

$(\beta_1+\beta_4)V_1+(\beta_1+\beta_2)V_2+(\beta_2+\beta_3)V_3 +(\beta_3+\beta_4)V_4$ =
$\begin{pmatrix}
0\\0\\0\\0
\end{pmatrix}$\\

let $\beta_5=\beta_1+\beta_4,~\beta_6=\beta_1+\beta_2\\
\beta_7=\beta_2+\beta_3~ and ~\beta_8=\beta_3+\beta_4$, then,\\

$\beta_5 V_1+ \beta_6  V_2+\beta_7 V_3+ \beta_8 V_4$ =
$\begin{pmatrix}
0\\0\\0\\0\\
\end{pmatrix}$\\
Since we know that $V_1,V_2,V_3,V_4$ are linearly independent then \\ $\beta_5=\beta_6=\beta_7=\beta_8=0$\\
Recall that,
$\beta_v=\beta_1+\beta_2,~ \beta_5=\beta_1+\beta_4,~\beta_{vi}=\beta_1+\beta_2\\
\beta_7=\beta_2+\beta_3~ and ~\beta_8=\beta_3+\beta_4$\\\\
so, 
$\beta_1+\beta_4=0 ~ ~ ~ ~ ~eq(i) \\
\beta_1+\beta_2=0 ~ ~ ~ ~ ~ ~ ~ ~eq(ii)\\
\beta_2+\beta_3=0  ~ ~ ~ ~ ~ ~ ~ eq(iii)\\
\beta_3+\beta_4=0  ~ ~ ~ ~ ~ ~ ~eq(iv)$

Subtract eq(ii) from eq(i) and eq(iii) from eq(iv)\\
then the set of vector  {$V_1+V_2,V_2,V_3,V_4$} are linearly independent then 
$\beta_4 - \beta_2=0\\
\beta_4-\beta_2=0$\\
Both equation gotten is the same. So, choose $\beta_2 = 1$\\
then, $\beta_4-1=0 \implies \beta_4=1$\\
 From eq(iii)$\beta_2+\beta_3=0 \implies \beta_1=-1$
 Therefore, $\beta_1=-1, \beta_2=1, \beta_3=-1 ~and ~beta_4=1$\\
 Hence, the vector $(V_1+V_2),(V_2+V_3),(V_3+V_4)$ and $(V_4+V_1)$ are \pmb{linearly dependent}.
\end{enumerate}
\end{enumerate}
\end{enumerate}
\end{document}
