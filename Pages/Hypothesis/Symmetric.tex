  \documentclass[45pt]{article}
\usepackage{blindtext}
\usepackage[english]{babel}
\usepackage[letterpaper,total={6in,8in},top=2cm, bottom=2cm, left=3cm,right=3cm,marginparwidth=1.75cm]{geometry}
\usepackage{mathptmx}
\usepackage[10pt]{moresize}
% Useful packages
\usepackage{amsmath,amssymb}
\usepackage{graphicx}
\usepackage[colorlinks=true, allcolors=blue]{hyperref}
\usepackage{amsfonts}
\usepackage{mathptmx}
\usepackage{anyfontsize}
\usepackage{t1enc}
\usepackage[utf8]{inputenc}

\title{GROUP 1}
\author{Project 1}

\begin{document}
\begin{titlepage}
    \begin{center}
        {\fontsize{70}{70}\selectfont MAT 212}
        
        \vspace{1cm}
        \Huge
        \textbf{Linear Algebra}
        
        \vspace{0.5cm}
        \LARGE
        Vector Space
        
        \vspace{0.5cm}
        {\fontsize{50}{50}\selectfont Group 1}
        \vspace{0.6cm}        
            
        A project providing solution to  different problems on \\vector Space.
        \vspace{0.8cm}
            
        \includegraphics[width=0.4\textwidth]{UI logo.png}

        \vspace{1.0cm}
        
        \Large{
         Mathematics (BSc. and B.Ed)\\
        University of Ibadan\\
        Nigeria.\\
        21/04/2023}
            
    \end{center}
\end{titlepage}
\begin{center}
\LARGE
\begin{tabular} { |p {1 cm}| p {5 cm} | p {2 cm} | p {5 cm} | p{2.2cm}|}

    \hline
    \multicolumn{5} { | c | }{\LARGE{Contributors}}\\
    \hline
    Sn & Names & Matric & Skill &Signature\\
    \hline
1& Oladejo abdullahi Titlayo & 221382 & maths and webdev Tutor&\\
\hline
 2& Ladoja Abimifoluwa Jemimah & 222667 & copy writing&\\
 \hline
3 & Oladapo Testimony Emmanuel & 222678& NFA&\\
  \hline
 4 & Akinsola, Qudus Oladimeji & 221375 & Graphics Design&\\
\hline
 5&OLUWADARA ADEOLUWA & 222691 & Product Management&\\
  \hline
  4&BLESSING ADEBAYO & 222634& GRAPHIC DESIGN&\\
  \hline
  7&Joshua Ebenezer Dayo & 222665& NFA&\\
  \hline
  8&Olanrewaju Bunmi Emmanuel &222681& Laptop Repair Specialist / content Creator&\\
  \hline
  9 & Falola Micheal Oluwatobi & 223769 &  web Design&\\
 \hline
 10&Oladipupo Joy Oluwatomisin & 230179 & entrepreneur&\\
  \hline
11& Olatunde Victor Olubiyi & 222686 &  NFA&\\
   
    \hline
 \end{tabular}
 \end{center}
 \vspace{2.0cmq}


\section{\Huge{Project 5}}\
\paragraph{\Huge{Exampe of Symmetric Tensor are :}}
\begin{enumerate}
    \item \Large{Symmetric Tensor of order 3 defined on $\mathbb{R}^2$ written as $ S^3(\mathbb{R}^2)$}\\
     $ S^3(\mathbb{R}^2)$=
$\begin{Bmatrix}
  A: A=
   \begin{bmatrix}
     \begin{pmatrix}
     a & b & \\
     b & d
     \end{pmatrix}
     &
      \begin{pmatrix}
     b & d & \\
     d & c
     \end{pmatrix}
   \end{bmatrix}$,
a,b,c,d $\in \mathbb{R}
     \end{Bmatrix}$

\item  \Large{Symmetric Tensor of order 3 defined on $\mathbb{R}^3$ written as $ S^3(\mathbb{R}^3)$}\\

$ S^3(\mathbb{R}^3)$=$\begin{Bmatrix}
  A: A=
   \begin{bmatrix}
     \begin{pmatrix}
     a & d & e\\
     d & h & f\\
     e & f & g\\
     \end{pmatrix}
     &
       \begin{pmatrix}
     d & h & f\\
     h & b & i\\
     f & i & j\\
     \end{pmatrix}
     &
      \begin{pmatrix}
     e & f & g\\
     f & i & j\\
     g & j & c\\
     \end{pmatrix}
   \end{bmatrix}
     \end{Bmatrix}$\\
     a,b,c,...j $\in \mathbb{R}$ 
\end{enumerate}


\section{Project Tensor}
\Large{Symmetric Tensor of order 3 defined on $\mathbb{R}^2$ written as $ S^3(\mathbb{R}^2)$}\\

$V=\begin{Bmatrix}  
  A: A=
 \begin{bmatrix}
  \begin{pmatrix} a_0&a_1\\ a_1&a_2\\
  \end{pmatrix}
  \begin{pmatrix}
  a_1&a_2\\
  a_2& a_3 \\
\end{pmatrix}
  \end{bmatrix}
  a_i \in \mathbb{R}
\end{Bmatrix}\\
$
\newcommand\fontsizeXii{\fontsize{12pt}{12pt}\selectfont}


\pmb{Basis is} $\begin{Bmatrix}
{\fontsizeXii
    \begin{bmatrix}
    \begin{pmatrix}
    1&0\\
    0&0\\
  \end{pmatrix}
    \begin{pmatrix}
    0&0\\
    0&0 \\
  \end{pmatrix}
\end{bmatrix},

\begin{bmatrix}
  \begin{pmatrix}
  0&1\\
  1&0\\
\end{pmatrix}
  \begin{pmatrix}
  1&0\\
  0&0 \\
\end{pmatrix}
\end{bmatrix},

\begin{bmatrix}
  \begin{pmatrix}
  0&0\\
  0&1\\
\end{pmatrix}
  \begin{pmatrix}
  0&1\\
  1&0 \\
\end{pmatrix}
\end{bmatrix},

\begin{bmatrix}
  \begin{pmatrix}
  0&0\\
  0&0\\
\end{pmatrix}
  \begin{pmatrix}
  0&0\\
  0&1 \\
\end{pmatrix}
\end{bmatrix}}
\end{Bmatrix}$


\section{Project 2}
\Large{Consider two bases of $\mathbb{R}^3$}
$\mathcal{B}_1$=
$\begin{Bmatrix}
    \begin{pmatrix}
        1\\1\\1
    \end{pmatrix}&
    
\begin{pmatrix}
        1\\1\\0
\end{pmatrix}&
\begin{pmatrix}
        1\\0\\0
\end{pmatrix}
\end{Bmatrix}$ and 

$\mathcal{B}_2=
\begin{Bmatrix}
    \begin{pmatrix}
        1\\1\\-1
    \end{pmatrix}&
    
\begin{pmatrix}
        1\\-1\\0
\end{pmatrix}&
\begin{pmatrix}
        2\\0\\0
\end{pmatrix}
\end{Bmatrix}$


Observe that 
$\begin{pmatrix} 1\\1 \end{pmatrix} and 
\begin{pmatrix} 1\\-1 \end{pmatrix}$ are linearly independent, since the vectors are in $\mathbb{R}2$ 
then $\begin{pmatrix} 1\\1 \end{pmatrix}$ and 
$\begin{pmatrix} 1\\-1 \end{pmatrix}$ spanned  $\mathbb{R}^2$ \\

\pmb{Hence they form a basis for the vector space defined above.}\\\\

For us to define the mapping we need to find the transformation of $\begin{pmatrix} a\\b \end{pmatrix} \forall_{a,b} \in \mathbb{R}$\\

Firstly, Let us find the coordinate of $\begin{pmatrix} a\\b \end{pmatrix}$ with respect to
$\begin{Bmatrix}
  \begin{pmatrix} 1\\1 \end{pmatrix} &
\begin{pmatrix} 1\\-1 \end{pmatrix} 
\end{Bmatrix}$

Let $\alpha ~~ and ~\beta$ be the coordinate then,\\
$\alpha \begin{pmatrix} 1\\1 \end{pmatrix}$ +
$\beta \begin{pmatrix} 1\\-1 \end{pmatrix} =\begin{pmatrix} a\\b \end{pmatrix}$
$\begin{pmatrix} \alpha+\beta\\ \alpha - \beta \end{pmatrix}=\begin{pmatrix} a\\b \end{pmatrix}$\\
$ \alpha+\beta =a~...eqI$\\
$ \alpha - \beta =b...eqII$\\
add eqI and eqII then we will have \\
$2 \alpha =a+b \implies \alpha =\frac{a+b}{2}$\\\\
From eqI, $ \alpha+\beta =a~...eqI$\\
so, $\frac{a+b}{2} +\beta =a$\\
$\beta =a-\frac{a+b}{2} \implies \beta = \frac{2a-a-b}{2}$
$\beta =\frac{a-b}{2}$\\
So, 

$\begin{pmatrix} a\\b \end{pmatrix} =\frac{a+b}{2} \begin{pmatrix} 1\\1 \end{pmatrix} + $
$\frac{a-b}{2} \begin{pmatrix} 1\\-1 \end{pmatrix}$\\\\
Since L is a mapping then,\\
$L\begin{pmatrix} \begin{pmatrix} a\\b \end{pmatrix} \end{pmatrix} =$
$L\begin{pmatrix} \frac{a+b}{2} \begin{pmatrix} 1\\1 \end{pmatrix} \end{pmatrix}$
+ 
$L\begin{pmatrix} \frac{a-b}{2} \begin{pmatrix} 1\\-1 \end{pmatrix} \end{pmatrix}$\\\\
$\implies \frac{a+b}{2} L\begin{pmatrix} \begin{pmatrix} 1\\1 \end{pmatrix} \end{pmatrix}$
+ 
$\frac{a-b}{2}L\begin{pmatrix}  \begin{pmatrix} 1\\-1 \end{pmatrix} \end{pmatrix}$ \\\\
But the transformation of $\begin{pmatrix} 1\\1 \end{pmatrix} ~and ~ \begin{pmatrix} 1\\-1 \end{pmatrix}$ are given\\
Therefore,\\
$L\begin{pmatrix} \begin{pmatrix} a\\b \end{pmatrix} \end{pmatrix} =$
$ \frac{a+b}{2} \begin{pmatrix} -1\\ 1\\2\\3 \end{pmatrix}$
+ 
$\frac{a-b}{2} \begin{pmatrix} 2\\0\\2\\3 \end{pmatrix}$
$\implies L\begin{pmatrix} \begin{pmatrix} a\\b \end{pmatrix} \end{pmatrix} =$
$ \begin{pmatrix} \frac{-a-b}{2} \\ \frac{a+b}{2} \\a+b\\ \frac{3a+3b}{2}  \end{pmatrix}$
+ 
$ \begin{pmatrix} \frac{2a-2b}{2}\\0\\a-b\\\frac{3a-3b}{2} \end{pmatrix}$\\\\\\
$\implies L\begin{pmatrix} \begin{pmatrix} a\\b \end{pmatrix} \end{pmatrix} =$
$ \begin{pmatrix} \frac{-a-b+2a-2b}{2} \\ \frac{a+b}{2} \\a+b+a-b\\ \frac{3a+3b+3a-3b}{2}  \end{pmatrix}$
= 
$ \begin{pmatrix} \frac{a-3b}{2}\\\frac{a+b}{2}\\2a\\\frac{a}{2} \end{pmatrix}$\\\\
$\implies L\begin{pmatrix} \begin{pmatrix} a\\b \end{pmatrix} \end{pmatrix} = \begin{pmatrix} \frac{a-3b}{2}\\\frac{a+b}{2}\\2a\\3a \end{pmatrix}$
\\\\\\\\
Hence the linear transformation can be defined as follow\\\\
$ L\begin{pmatrix} \begin{pmatrix} a\\b \end{pmatrix} \end{pmatrix} = \begin{pmatrix} \frac{a-3b}{2}\\\frac{a+b}{2}\\2a\\3a \end{pmatrix}$


\end{document}
